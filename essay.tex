% Please do not change the document class
\documentclass{scrartcl}

% Please do not change these packages
\usepackage[hidelinks]{hyperref}
\usepackage[none]{hyphenat}
\usepackage{setspace}
\doublespace

% You may add additional packages here
\usepackage{amsmath}

% Please include a clear, concise, and descriptive title
\title{Evidence-Based Practice for Designing Accessible Games: How Can Mainstream Games Accommodate for the Visually Impaired?}

% Please do not change the subtitle
\subtitle{COMP160 - Software Engineering}

% Please put your student number in the author field
\author{1608351}

\begin{document}

\maketitle

\abstract{Please include an abstract of at most 100 words (these do not count towards your word count).}

\section{Introduction}

{Developing games to meet the many varying cognitive, physical, and visual needs of the population is a complicated and often expensive task. Whilst other software, such as office and web applications,  have been adapted successfully for the visually impaired, accessibility in the gaming industry is comparatively poor.

The reliance of mainstream games on impressive visuals makes adaptation for the visually impaired far more complex than other applications. However, several creative technologies and strategies have been designed to improve games for the visually impaired;  using music to signify transitions \cite{1}, echoes to describe 3D spaces \cite{5}, specially designed gaming devices \cite{2} \cite{8}, and the conversion of images into sounds \cite{9}.

Within software engineering, developers often face tight deadlines which may prevent accessibility from being prioritized in the engineering of games \cite{3} \cite{10} . This paper aims to explore obstacles facing the development of accessible games and how mainstream games can learn from software designed specifically for the visually impaired.}


\section{Case Studies}

\textbf{AudioQuake:}
 
Quake \cite{4} is one of the first FPS mainstream games to be adapted to suit the visually impaired, in the form of AudioQuake, which uses an auditory interface \cite{5}. The creators were keen to not only enable basic game play, but ensure blind users could participate online and in the building of their own level maps. Adapting Quake into an accessible game for the visually impaired involved several stages of development, beginning with designing low-level requirements necessary to implement an auditory interface. Audio needed to be improved to create more realistic cues for the visually impaired, such as indicating room size using echoes. AudioQuake is part of the AGRIP project \cite{6} which aims for games, not only playable by the disabled, but that also allow evenly matched play with or against able-bodied users, the ability to create their own map levels and to adjust game settings. AudioQuake was built with LDL (Level Description Language), an XML programming language designed specifically to enable developers to describe 3D spaces and for visually impaired players to be able to create 3D levels. A vital part of the AGRIP Project is community contribution, particularly feedback from disabled-users. During development, AudioQuake was regularly play-tested at conventions, such as Sight Village, and conversations between users on the mailing list were encouraged for greater insight into the game's effectiveness.

\textbf{AccTrace:}

Although considerable research and proposals for improving Software Engineering for accessibility exist, developers rarely understand how to code accessible systems. AccTrace \cite{12} is a Case Tool designed to use an ontology, a method for showing a group of concepts and the relationships between them, to propose and define a technical procedure for implementing accessible gaming. AccTrace was also designed to enable traceability, allowing developers to ascertain the origin of certain elements within a systems design. \cite{12}

\textbf{A Walk in the Park:}

\section{Software Engineering for Accessibility in Mainstream Games}
In the  1980's, computer software applications first started to be adapted for the disabled. However, mainstream video games remain overwhelming inaccessible \cite{13}, \textit{`vital considerations for improving accessibility and usability are lacking in the software engineering process."} \cite{15}

\section{Obstacles to Accessible Software}

1.	Valuing of Best Practices in Software Engineering: in academia, time constraints/pressures, focus on product being delivered asap

2.	Valuing of accessibility in software engineering, and specifically value of gaming for the visually impaired:

3.	Financial and Time Constraints: 
The game industry "the efforts of game accessibility must have a realistic financial grounding, otherwise they risk not become implemented in mainstream games." "to get mainstream games to be accessible to as many as possible we need first to resolve the financial issues, which are related to the time and effort accessibility development takes, and the increased number of sales you get by doing it." \cite{14}

4.	Evenly-matched games between able and non-able bodied players

5.	Standardised Guidelines for Accessible Games:

\section{Recommendations}

1.	Standardised Guidelines

2.	User Testing/Feedback for target group

3.	Multimodal User Interfaces: Not only to suit visually impaired but diverse range of needs

4.	Allowing users to make modifications to game: Not just basic playable, enable evenly-matched play against able-bodied users and the ability to adjust map levels, game settings, personalized characters (through use of audio/voice rather than appearance)

5.	Prioritization of Accessibility: Teaching of accessible game engineering best practices in higher-education, awareness of benefits of games for visually impaired.

6.	Make use of and develop further available software engineering tools for accessibility: LDL, AccTrace

7.	Games must still be games when adapted. Not just usable, but enjoyable

\section{Conclusion}

Application of accessibility during the development of games, and not only an afterthought


\bibliographystyle{ieeetran}
\bibliography{references}

\end{document}
