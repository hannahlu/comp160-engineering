% Please do not change the document class
\documentclass{scrartcl}

% Please do not change these packages
\usepackage[hidelinks]{hyperref}
\usepackage[none]{hyphenat}
\usepackage{setspace}
\doublespace

% You may add additional packages here
\usepackage{amsmath}

% Please include a clear, concise, and descriptive title
\title{Evidence-Based Practice for Designing Accessible Games: How Can Mainstream Games Accommodate for the Visually Impaired?}

% Please do not change the subtitle
\subtitle{COMP160 - Software Engineering}

% Please put your student number in the author field
\author{1608351}

\begin{document}

\maketitle

\abstract{Mainstream video games are rarely adequately adapted for the visually impaired. In comparison to the greater accessibility of web and office applications, this can be understood by the more complex visual and immersive nature of games. However, game developers have a responsibility to those with disabilities to make their games accessible. This paper explores best practices in software engineering for accessibility, focusing on three existing accessible software case studies; Quake \cite{4}, AccTrace \cite{12} and a Disability-Aware Software Engineering Model \cite{15}.}

\section{Introduction}

{Developing games to meet the many varying cognitive, physical, and visual needs of the population is a complicated and often expensive task. Whilst other software, such as office and web applications, have been adapted successfully for the visually impaired, accessibility in the gaming industry is comparatively poor \cite{13}.

The reliance of mainstream games on impressive visuals makes adaptation for the visually impaired far more complex than other applications. However, several creative technologies have been designed to improve games for the visually impaired; using music to signify transitions \cite{1}, echoes to describe 3D spaces \cite{5}, specially designed gaming devices \cite{2} \cite{8}, and the conversion of images into sounds \cite{9}. This paper aims to explore obstacles facing the development of accessible games and how mainstream games can learn from software designed specifically for the visually impaired.}

\section{Case Study: AudioQuake}

Quake \cite{4} is one of the first FPS mainstream games to be adapted to suit the visually impaired. AudioQuake uses an auditory interface \cite{5} and as part of the AGRIP project \cite{6}, which stresses the importance of adapted games not merely being ‘playable’ by those with disabilities, enables visually impaired users to create their own map levels and participate online. 

AudioQuake was built with LDL (Level Description Language), an XML programming language designed to enable developers to describe 3D spaces and visually impaired users to create 3D levels. In adapting Quake, audio was enhanced to create more realistic cues for the visually impaired, such as indicating room size using echoes. A vital part of the AGRIP Project is feedback from disabled-users and conversations between users on the mailing list were encouraged for greater insight into the game's effectiveness.

\section{Case Study: AccTrace}

Although considerable research and proposals to improve software engineering for accessibility exist, developers rarely understand how to code accessible software \cite{12} . AccTrace is a case tool designed to use an ontology to propose a system for implementing accessibility. AccTrace was also designed to enable traceability, allowing developers to determine the origin and effects of accessible requirements. The key concepts of the AccTrace tool include; having an accessibility specialist in the development team, specifying the accessibility requirements of the software from the outset, producing UML diagrams based on these and stating the technical implementation for each UML association and requirement \cite{12}.

\section{Case Study: Disability-aware Software Engineering Processing Model}

Nganji and Nggada \cite{15} have proposed a software engineering model, capable of adapting for a range of disabilities. Disabled-user personas are created to develop initial accessibility requirements, minimising the need for costly prototype testing. These personas should be developed with input from disability experts and disabled-users. Once the system has been coded and all individual components linked together, disabled-user testing should be performed to find errors and evaluate the effectiveness of the software \cite{15}.

\section{Obstacles to Software Engineering for Accessibility}

Minority group such as the visually impaired, are often overlooked, being too small a proportion of the market, to affect the profitability of games \cite{16}. Accessibility is rarely incorporated in game design from the start of the software engineering cycle \textit{`vital considerations for improving accessibility and usability are lacking in the software engineering process"} \cite{15}. Developers are under constant pressure to meet tight deadlines, and may not consider accessibility vital to development \cite{3} \cite{10}. 

Yet, research suggests engineering for accessibility from the start of development, helps produce stronger games for all users \cite{5}, as found with web applications, \textit{`websites are on average 35\% easier to use for everyone if they comply with accessibility standards [Disability Rights Commission 2004]"} \cite{5}. Benefits can be attributed to the increased personalisation capabilities and robust user interfaces necessary for accessible software. 

Game developers may also benefit from accessibility, \textit{`adding usability to a software development program can provide up to a one hundred-fold return on investment (Bias \& Mayhew, 1994)."} \cite{17} \cite{20}. Engineering for accessibility can highlight problems in development otherwise overlooked and decrease the level of user support required, due to greater user-modification capabilities and stronger user interfaces \cite{17}. Therefore, not only is accessibility a fundamental responsibility of game developers, just as all industries have a responsibility to adapt services and products for the disabled, engineering for accessibility may benefit all users. 

The costs of implementing accessibility is another obstacle, \textit{`the efforts of game accessibility must have a realistic financial grounding, otherwise they risk not become implemented in mainstream games."} \cite{14}. Whilst the resources necessary for accessible software do introduce further cost, these can be reduced through planning, use of best practices and applying accessibility from the start \cite{17}. 

\section{Standardised Guidelines for Accessible Games}

Accessible gaming lacks a set of standardised guidelines \cite{18}. Whilst game developers can take inspiration from existing web accessibility guidelines \cite{18}, video games require more complex information to be translated to visually impaired users. A set of guidelines, specifically for accessible games, is of need. This paper outlines a few recommendations for accessible engineering, as follows:

\begin{enumerate}
\item Game developers should make the main game, rather than a secondary version, accessible, \textit{`separate is not equal"} \cite{17}. When developers are under-pressure, the secondary game will receive less care and the cost of running two separate versions of the software will likely be greater \cite{17}.
\item A generalised attempt at accessibility is unlikely to be effective \cite{17}. Game developers need a clear understanding of the disabilities they are adapting for when designing accessible software, advice from disability experts should be sought. 
\item Disabled-user testing should be prioritised throughout the process \textit{`The user is the centre of this design and if the system does not meet their need, it will be irrelevant.} \cite{15}.
\item An accessibility specialist should be on the development team, responsible for defining and implementing accessibility requirements \cite{12}. However, all members of the team need an understanding of accessibility in relation to their discipline, and have a responsibility towards producing an accessible product \cite{17}.
\item Games should enable evenly-matched play between both able and non-able bodied users \cite{11}.
\item User Interfaces should incorporate multimodality, enabling information to be exchanged through different input modalities \cite{16}. Examples for the visually impaired, include; speech recognition, braille devices, gaze and body movements. Enabling greater user-modification of software settings is necessary to suit the varying needs of the visually impaired, and can also benefit non-disabled users \cite{15}.
\item Accessibility requirements should be incorporated from the outset, making use of existing technologies and tools such as, AccTrace, to simplify the process. \cite{12}
\item Once adapted, games should still be enjoyable and not merely usable by disabled players \cite{19}.
\end{enumerate}

\section{Conclusion}

Mainstream game developers have a responsibility to the visually impaired to make software accessible \cite{17}. Software engineering for accessibility is most effective when incorporated from the outset of development and in doing so, the robust nature of accessibility requirements, will likely benefit all users \cite{5}, \cite{17}, \cite{20}. There remains need for a set of standardised guidelines for accessible gaming, if accessibility is to become the norm in mainstream games \cite{18}. Potential recommendations are outlined in this paper, including; the presence of an accessibility specialist in development teams, consultation with disability experts, use of multimodal user interfaces and disabled user-testing.


\bibliographystyle{ieeetran}
\bibliography{references}

\end{document}
